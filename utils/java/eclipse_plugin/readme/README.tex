\documentclass[10pt]{article}
\usepackage[top=1cm, bottom=1cm, left=1cm, right=1cm]{geometry}

%opening
\title{Eclipse UMC Plugin Readme}
\author{Timotei Dolean}

\begin{document}

\maketitle

\newcounter{cnt}
\newcommand{\icnt}{ \stepcounter{cnt} \thecnt }

\section{Common prerequisites}
\begin{enumerate}
\item Download and install ``Eclipse for RCP and RAP Developers" (\textbf{http://eclipse.org/downloads/packages/eclipse-rcp-and-rap-developers/heliosr}) -
The download links are in the right. Please ensure you are downloading the \textbf{3.6} version,
otherwise the plugin will not work.
\item After launching Eclipse, go to the ``Help'' menu - Install new Software. \\
  Then, please check ``Group items by category" and ``Contact all update sites during install to find required software",
  in the bottom of the page.
\item Insert the link \textbf{http://download.eclipse.org/releases/helios} in the textbox in the top and press Enter.
  The list will be populated with some items.
\item From there select in the ``Modelling" category, \textbf{Xtext SDK}.
  Press next and finish the install.\\
  \textit{Note:} If you are prompted for any license agreements or certificates press Yes on all (if you agree).
\end{enumerate}

\section{User}
\subsection{Installing the plugin}
\begin{enumerate}
\item Install the plugin from (\textbf{http://eclipse.wesnoth.org/}), using the same sequence as in ``1.2 and 1.3"
\item Select \textbf{``Wesnoth UMC Plugin"} and press finish.
\end{enumerate}

\section{Developer}
\subsection{Setup the environment}
\begin{enumerate}
\item Checkout plugin's folder from the svn (http://svn.gna.org/svn/wesnoth/trunk/utils/java/) and select the following folders: ``eclipse\_plugin", ``org.wesnoth.wml'', ``org.wesnoth.wml.ui"
\item In Eclipse, right click in \textbf{Package Explorer/Project Navigator} and then select
 \textbf{Import - General - Existing projects into Workspace}
\item Select the path where you downloaded the java folder, and check all the 3 projects: ``eclipse\_plugin", ``org.wesnoth.wml'', ``org.wesnoth.wml.ui".
\item Build the projects.
\end{enumerate}

\subsection{Running the plugin}
After you've setup the environment and built the plugin you can run it.
\begin{enumerate}
\item Open the file plugin.xml
\item In the \textbf{Testing} section, select the desired method of launching the plugin
  (non-debug/debug mode).
\end{enumerate}

\section{Everybody}
\subsection{Using the plugin}
After you have your plugin installed(user) or running(developer), you can use its features.
But before of all, you must ``Setup the workspace". For this, go in the ``Wesnoth" menu and from 	
Should the working directory be empty, it will be computed automatically
based on wesnoth's executable

\subsubsection{Wizards}
To create a new \textbf{Campaign}, open the ``New..." menu (either from File - New menu, or right click in the
project navigator and select ``New..." ). After that select ``Wesnoth Campaign". Fill in the information needed,
and press finish. Your campaign project is created in the workspace. \\
To create a new \textbf{Scenario}, open the ``New..." menu, and select ``Wesnoth scenario".
Complete the information needed and press finish.

\subsubsection{Menus}
There are currently 2 types of menus: the context menus for different file/folder types and the toolbar menus.
\begin{description}
\item{\textbf{Project context menus}} - right click on the campaign projects created with the plugin\\
   {\it Wesnoth project report} - will show a simple report with the numer of maps, scenarios and units.

\item{\textbf{.cfg files context menus}} - right click on any .cfg file\\
   {\it Open scenario in game} - opens the selected file's scenario (if it contains one) in wesnoth \\
   {\it WML Tools} - provides some options for using the wmltools with the specified file
   (e.g. run wmllint against the file and see the output in the console) \\
   {\it Preprocessor} - provides ways of preprocessing and showing the result in an editor inside eclipse.

\item{\textbf{``maps" folder}} - right click on the "maps" folder\\
   {\it Import map} - Shows a file selection window that let's you select a .map file that will be copied in your campaign project.

\end{description}
\end{document}
